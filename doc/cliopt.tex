% This file is automatically generated; do not edit
\begin{lstlisting}
Syntax: exscript [options] exscript [hostname [hostname ...]]
      --account-pool FILE
                 Reads the user/password combination from the given file 
                 instead of prompting on the command line. The file may
                 also contain more than one user/password combination, in
                 which case the accounts are used round robin.
  -c, --connections NUM
                 Maximum number of concurrent connections.
                 NUM is a number between 1 and 20, default is 1
      --csv-hosts FILE
                 Loads a list of hostnames and definitions from the given file.
                 The first line of the file must have the column headers in the
                 following syntax:
                    hostname [variable] [variable] ...
                 where the fields are separated by tabs, "hostname" is the
                 keyword "hostname" and "variable" is a unique name under
                 which the column is accessed in the script.
                 The following lines contain the hostname in the first column,
                 and the values of the variables in the following columns.
  -d, --define PAIR
                 Defines a variable that is passed to the script.
                 PAIR has the following syntax: <STRING>=<STRING>.
      --default-domain STRING
                 The IP domain name that is used if a given hostname has no 
                 domain appended.
  -e, --execute EXSCRIPT
                 Interpretes the given string as the script.
      --hosts FILE
                 Loads a list of hostnames from the given file (one host per
                 line).
  -i, --non-interactive
                 Do not ask for a username or password.
  -l, --logdir DIR
                 Logs any communication into the directory with the given name.
                 Each filename consists of the hostname with "_log" appended.
                 Errors are written to a separate file, where the filename
                 consists of the hostname with ".log.error" appended.
      --no-echo
                 Turns off the echo, such that the network activity is no longer
                 written to stdout.
                 This is already the default behavior if the -c option was given
                 with a number greater than 1.
  -n, --no-authentication
                 When given, the authentication procedure is skipped. Implies -i.
      --no-auto-logout
                 Do not attempt to execute the exit or quit command at the end
                 of a script.
      --no-prompt
                 Do not wait for a prompt anywhere. Note that this will also
                 cause Exscript to disable commands that require a prompt, such
                 as 'extract'.
      --no-initial-prompt
                 Do not wait for a prompt after sending the password.
      --no-strip
                 Do not strip the first line of each response.
      --overwrite-logs
                 Instructs Exscript to overwrite existing logfiles. The default
                 is to append the output if a log already exists.
  -p, --protocol STRING
                 Specify which protocol to use to connect to the remote host.
                 STRING is one of: telnet ssh
                 The default protocol is telnet.
      --retry-login NUM
                 Defines the number of retries on login failure. Default is 0.
      --sleep TIME
                 Waits for the specified time before running the script.
                 TIME is a timespec as specified by the 'sleep' Unix command.
      --ssh-auto-verify
                 Automatically confirms the 'Host key changed' SSH error 
                 message with 'yes'. Highly insecure and not recommended.
      --ssh-key FILE
                 Specify a key file that is passed to the SSH client.
                 This is equivalent to using the -i parameter with ssh.
  -v, --verbose NUM
                 Print out debug information about the network activity.
                 NUM is a number between 0 (min) and 5 (max)
  -V, --parser-verbose NUM
                 Print out debug information about the Exscript parser.
                 NUM is a number between 0 (min) and 5 (max)
      --version  Prints the version number.
  -h, --help     Prints this help.
\end{lstlisting}
